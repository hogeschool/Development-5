\section{Assessment}

The exam consists of a series of exercises where students, given partial
code and the desired state transitions, are requested to fill in the
missing code that matches the given state transitions.
\\
The exercises are split as follows: 
\begin{itemize}
\item Part 1, 25\% of the exercises, requires students to complete a model implementation; 
\item Part 2, 25\% of the exercises, requires students to complete a controller implementation; 
\item Part 3, 50\% of the exercises, requires students to complete a view implementation.
\end{itemize}

For example the exam could be made of 8 exercises: 2 about the model, 2
about the controller and 4 about the view.

This reflects the relatively higher emphasis on interaction that modern
distributed applications are showing. This leads development time to be
split non-uniformly across the architectural elements.

\subsection*{Scoring}

The exam results in a full grade (from 0 to 10). Each exercise awards
one single point, if correctly completed. The grade is computed as the
percentage of points obtained, divided by 10. Students who score at
least 55\% of the total points will get a passing grade. For example, if
a student completes 5 exercises out of 8, this means obtaining 5 points
out of 8, which means 62,5\% and corresponds thus to a 6,25 (62,5/10).

\subsection*{Matrix}
The exam covers all learning goals.
\begin{center}
\begin{tabular}{ |c|c|c|c| } 
\hline 
Exam part & Part 1 &  Part 2  & Part 3 \\
\hline
PR\_M  & V &&\\
PR\_C  & &V& \\
PR\_V & & & V \\
\hline
\end{tabular}
\end{center}

\subsection{Retake}
 If the exam is not passed, then it will need to be retaken during the current schoolyear.
The retake will be scheduled at the end of the following period.



	
